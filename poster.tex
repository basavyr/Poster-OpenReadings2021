\documentclass[final]{beamer}

% ====================
% Packages
% ====================

\usepackage[T1]{fontenc}
\usepackage{lmodern}
\usepackage[size=custom,width=120,height=72,scale=1.0]{beamerposter}
\usetheme{gemini}
\usecolortheme{gemini}
\usepackage{graphicx}
\usepackage{booktabs}
\usepackage{tikz}
\usepackage{xcolor}
\usepackage{pgfplots}
\pgfplotsset{compat=1.7}

% ====================
% Lengths
% ====================

% If you have N columns, choose \sepwidth and \colwidth such that
% (N+1)*\sepwidth + N*\colwidth = \paperwidth
\newlength{\sepwidth}
\newlength{\colwidth}
\setlength{\sepwidth}{0.025\paperwidth}
\setlength{\colwidth}{0.3\paperwidth}

\newcommand{\separatorcolumn}{\begin{column}{\sepwidth}\end{column}}

% ====================
% Title
% ====================

\title{SINGLE-PARTICLE MOTION IN A WOBBLING NUCLEUS \\ A CASE-STUDY FOR ODD-MASS ISOTOPES}

\author{Robert Poenaru \inst{1,2} | \texttt{robert.poenaru@protonmail.ch}}

%\institute[shortinst]{\inst{1} Doctoral School of Physics, University of Bucharest, Bucharest, Romania \samelineand \inst{2} Department of Theoretical Physics, Horia-Hulubei National Institute of Nuclear Physics and Engineering, Bucharest-Magurele, Romania}

\institute[shortinst]{\inst{1} Doctoral School of Physics, University of Bucharest, Bucharest, Romania \\ \inst{2} Department of Theoretical Physics, Horia-Hulubei National Institute of Nuclear Physics and Engineering, Bucharest-Magurele, Romania}

% ====================
% Body
% ====================

\begin{document}

\addtobeamertemplate{headline}{} 
{\begin{tikzpicture}[remember picture, overlay]
     \node [anchor=north east, inner sep=0.9cm]  at (current page.north east)
     {\includegraphics[height=11cm]{./images/logo.png}};
     \node [anchor=north west, inner sep=0.9cm]  at (current page.north west)
     {\includegraphics[height=11cm]{./images/uniLogo.png}};
  \end{tikzpicture}}


\begin{frame}[t]
\begin{columns}[t]
\separatorcolumn

\begin{column}{\colwidth}

  \begin{block}{Introduction}

In the present work, a description of the collective motion which occurs in strongly-deformed odd-mass nuclei (the mass number $A$ is odd) is done, using the \emph{Particle Rotor Model} (PRM) \cite{bohr1998nuclear}. Within this framework, the total nuclear system consists of an even-even core and a \emph{valence} nucleon which is moving in a quadrupole deformed mean field, generated by the core.

The strength of that potential is crucial in the description of the wobbling spectrum of a nucleus. In the present work, an analysis of the potential strength that characterizes the coupling between the core and the odd-nucleon is made, with the help of a deformed Nilsson potential in the total Hamiltonian of the system. A study of the coupling term is performed for different isotopes in which wobbling motion is known to occur.

  \end{block}

  \begin{block}{Wobbling Motion}
  
\begin{itemize}
    \item The wobbling phenomenon in nuclei implies a \emph{precession} of the total angular momentum combined with an \emph{oscillation} of its projection onto the rotation axis.
    \item Triaxial nuclei are objects with all three moments of inertia associated with the principal axes different in magnitude, making it possible for rotation to occur around all three axes. This results in a rich rotational spectrum with a collective character.
\end{itemize}

  \end{block}
  
    \begin{block}{Wobbling Motion - Odd-A Case}
  
  \end{block}

\end{column}

\separatorcolumn

\begin{column}{\colwidth}

  \begin{block}{Single-Particle Deformed Potential}

  \end{block}
  
\end{column}

\separatorcolumn

\begin{column}{\colwidth}

  \begin{block}{A block containing some math}

    \heading{A heading inside a block}
  One.

    \heading{Another heading inside a block}
 One.

  \end{block}

  \begin{block}{Results}
    Some results.
    \begin{table}
      \centering
      \begin{tabular}{l r r c}
        \toprule
        \textbf{First column} & \textbf{Second column} & \textbf{Third column} & \textbf{Fourth} \\
        \midrule
        Foo & 13.37 & 384,394 & \alpha \\
        Bar & 2.17 & 1,392 & \beta \\
        Baz & 3.14 & 83,742 & \delta \\
        Qux & 7.59 & 974 & \gamma \\
        \bottomrule
      \end{tabular}
      \caption{A table caption.}
    \end{table}
  \end{block}
  
    \begin{block}{Conclusions}
    Some conclusions.
    \begin{table}
      \centering
      \begin{tabular}{l r r c}
        \toprule
        \textbf{First column} & \textbf{Second column} & \textbf{Third column} & \textbf{Fourth} \\
        \midrule
        Foo & 13.37 & 384,394 & \alpha \\
        Bar & 2.17 & 1,392 & \beta \\
        Baz & 3.14 & 83,742 & \delta \\
        Qux & 7.59 & 974 & \gamma \\
        \bottomrule
      \end{tabular}
      \caption{A table caption.}
    \end{table}
  \end{block}

  \begin{block}{References}

    \nocite{*}
    \footnotesize{\bibliographystyle{plain}\bibliography{poster}}

  \end{block}

\end{column}

\separatorcolumn

\end{columns}

\end{frame}

\end{document}