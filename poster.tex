\documentclass[final]{beamer}

% ====================
% Packages
% ====================

\usepackage[T1]{fontenc}
\usepackage{lmodern}
\usepackage[size=custom,width=120,height=72,scale=1.0]{beamerposter}
\usetheme{gemini}
\usecolortheme{gemini}
\usepackage{graphicx}
\usepackage{booktabs}
\usepackage{tikz}
\usepackage{xcolor}
\usepackage{pgfplots}
\pgfplotsset{compat=1.7}

% ====================
% Lengths
% ====================

% If you have N columns, choose \sepwidth and \colwidth such that
% (N+1)*\sepwidth + N*\colwidth = \paperwidth
\newlength{\sepwidth}
\newlength{\colwidth}
\setlength{\sepwidth}{0.025\paperwidth}
\setlength{\colwidth}{0.3\paperwidth}

\newcommand{\separatorcolumn}{\begin{column}{\sepwidth}\end{column}}

% ====================
% Title
% ====================

\title{SINGLE-PARTICLE MOTION IN A WOBBLING NUCLEUS \\ A CASE-STUDY FOR ODD-MASS ISOTOPES}

\author{Robert Poenaru \inst{1,2} \\ \texttt{robert.poenaru@protonmail.ch}}

%\institute[shortinst]{\inst{1} Doctoral School of Physics, University of Bucharest, Bucharest, Romania \samelineand \inst{2} Department of Theoretical Physics, Horia-Hulubei National Institute of Nuclear Physics and Engineering, Bucharest-Magurele, Romania}

\institute[shortinst]{\inst{1} Doctoral School of Physics, University of Bucharest, Bucharest, Romania \\ \inst{2} Department of Theoretical Physics, Horia-Hulubei National Institute of Nuclear Physics and Engineering, Bucharest-Magurele, Romania}

% ====================
% Body
% ====================

\begin{document}

\addtobeamertemplate{headline}{} 
{\begin{tikzpicture}[remember picture, overlay]
     \node [anchor=north east, inner sep=0.9cm]  at (current page.north east)
     {\includegraphics[height=11cm]{./images/logo.png}};
     \node [anchor=north west, inner sep=0.9cm]  at (current page.north west)
     {\includegraphics[height=11cm]{./images/uniLogo.png}};
  \end{tikzpicture}}


\begin{frame}[t]
\begin{columns}[t]
\separatorcolumn

\begin{column}{\colwidth}

  \begin{block}{Abstract}

The wobbling phenomenon in nuclei, which implies a precession of the total angular momentum combined with an oscillation of its projection onto the rotation axis, is analyzed within the Particle Rotor Model for odd-mass nuclei. Triaxial nuclei are objects with all three moments of inertia associated with the principal axes different in magnitude, making it possible for rotation to occur around all three axes. This results in a rich rotational spectrum with a collective character. Interpretation of the wobbling motion in odd-mass nuclei is usually done through a particle-rotor coupling, where an even-even triaxial is coupled to an odd-$j$ nucleon which is said to be moving in a deformed quadrupole mean-field generated by the core itself. The strength of that potential is crucial in the description of the wobbling spectrum of a nucleus. In the present work, an analysis of the potential strength that characterizes the coupling between the core and the odd-nucleon is made, with the help of a deformed Nilsson potential in the total Hamiltonian of the system. A study of the coupling term is performed for different isotopes in which wobbling motion is known to occur.

  \end{block}

  \begin{block}{A block containing a list}

    Nam vulputate nunc felis, non condimentum lacus porta ultrices. Nullam sed
    sagittis metus. Etiam consectetur gravida urna quis suscipit.

    \begin{itemize}
      \item \textbf{Mauris tempor} risus nulla, sed ornare
      \item \textbf{Libero tincidunt} a duis congue vitae
      \item \textbf{Dui ac pretium} morbi justo neque, ullamcorper
    \end{itemize}

    Eget augue porta, bibendum venenatis tortor.

  \end{block}

\end{column}

\separatorcolumn

\begin{column}{\colwidth}

  \begin{block}{A block containing an enumerated list}

    Vivamus congue volutpat elit non semper. Praesent molestie nec erat ac
    interdum. In quis suscipit erat. \textbf{Phasellus mauris felis, molestie
    ac pharetra quis}, tempus nec ante. Donec finibus ante vel purus mollis
    fermentum. Sed felis mi, pharetra eget nibh a, feugiat eleifend dolor. Nam
    mollis condimentum purus quis sodales. Nullam eu felis eu nulla eleifend
    bibendum nec eu lorem. Vivamus felis velit, volutpat ut facilisis ac,
    commodo in metus.

    \begin{enumerate}
      \item \textbf{Morbi mauris purus}, egestas at vehicula et, convallis
        accumsan orci. Orci varius natoque penatibus et magnis dis parturient
        montes, nascetur ridiculus mus.
      \item \textbf{Cras vehicula blandit urna ut maximus}. Aliquam blandit nec
        massa ac sollicitudin. Curabitur cursus, metus nec imperdiet bibendum,
        velit lectus faucibus dolor, quis gravida metus mauris gravida turpis.
      \item \textbf{Vestibulum et massa diam}. Phasellus fermentum augue non
        nulla accumsan, non rhoncus lectus condimentum.
    \end{enumerate}

  \end{block}

  \begin{block}{Fusce aliquam magna velit}

    Et rutrum ex euismod vel. Pellentesque ultricies, velit in fermentum
    vestibulum, lectus nisi pretium nibh, sit amet aliquam lectus augue vel
    velit. Suspendisse rhoncus massa porttitor augue feugiat molestie. Sed
    molestie ut orci nec malesuada. Sed ultricies feugiat est fringilla
    posuere.

    \begin{figure}
      \centering
      \begin{tikzpicture}
        \begin{axis}[
            scale only axis,
            no markers,
            domain=0:2*pi,
            samples=100,
            axis lines=center,
            axis line style={-},
            ticks=none]
          \addplot[red] {sin(deg(x))};
          \addplot[blue] {cos(deg(x))};
        \end{axis}
      \end{tikzpicture}
      \caption{Another figure caption.}
    \end{figure}

  \end{block}

\end{column}

\separatorcolumn

\begin{column}{\colwidth}

  \begin{block}{A block containing some math}

    Nullam non est elit. In eu ornare justo. Maecenas porttitor sodales lacus,
    ut cursus augue sodales ac.

    $$
    \int_{-\infty}^{\infty} e^{-x^2}\,dx = \sqrt{\pi}
    $$

    Interdum et malesuada fames $\{1, 4, 9, \ldots\}$ ac ante ipsum primis in
    faucibus. Cras eleifend dolor eu nulla suscipit suscipit. Sed lobortis non
    felis id vulputate.

    \heading{A heading inside a block}

    Praesent consectetur mi $x^2 + y^2$ metus, nec vestibulum justo viverra
    nec. Proin eget nulla pretium, egestas magna aliquam, mollis neque. Vivamus
    dictum $\mathbf{u}^\intercal\mathbf{v}$ sagittis odio, vel porta erat
    congue sed. Maecenas ut dolor quis arcu auctor porttitor.

    \heading{Another heading inside a block}

    Sed augue erat, scelerisque a purus ultricies, placerat porttitor neque.
    Donec $P(y \mid x)$ fermentum consectetur $\nabla_x P(y \mid x)$ sapien
    sagittis egestas. Duis eget leo euismod nunc viverra imperdiet nec id
    justo.

  \end{block}

  \begin{block}{Nullam vel erat at velit convallis laoreet}

    Class aptent taciti sociosqu ad litora torquent per conubia nostra, per
    inceptos himenaeos. Phasellus libero enim, gravida sed erat sit amet,
    scelerisque congue diam. Fusce dapibus dui ut augue pulvinar iaculis.

    \begin{table}
      \centering
      \begin{tabular}{l r r c}
        \toprule
        \textbf{First column} & \textbf{Second column} & \textbf{Third column} & \textbf{Fourth} \\
        \midrule
        Foo & 13.37 & 384,394 & \alpha \\
        Bar & 2.17 & 1,392 & \beta \\
        Baz & 3.14 & 83,742 & \delta \\
        Qux & 7.59 & 974 & \gamma \\
        \bottomrule
      \end{tabular}
      \caption{A table caption.}
    \end{table}

  \end{block}

  \begin{block}{References}

    \nocite{*}
    \footnotesize{\bibliographystyle{plain}\bibliography{poster}}

  \end{block}

\end{column}

\separatorcolumn
\end{columns}
\end{frame}

\end{document}
